\documentclass[a4paper, 10pt]{article}
\usepackage[utf8]{inputenc}
\usepackage[slovene]{babel}

\begin{document}
\title{Domača naloga 1 pri predmetu ITAP}
\author{Anej Rozman}
\date{}
\maketitle

\begin{subsection}*{Naloga 1}

    \begin{subsubsection}*{a)}
        Dobil sem dve celi "stevili in dve "stevili, ki sta prakti"cno $0$. Skratka 
        model se je popolnoma prilegal podatkom, ker so bili simulirani z linearno funkcijo. 
        \newline
        Koeficienti: $[ 1,   5, -4.00027234\cdot 10^{-15}, \ -7.21644966\cdot10^{-16}]$
    \end{subsubsection}

    \begin{subsubsection}*{b)}
        Pri tem delu naloge sem samo dodal stolpec enic matriki $X$, saj glede na to da obravnavamo 
        podatke, ki so popolnoma linearno kolerirani med sabo, je to na"cin, da se bo napovedni model
        (skoraj) popolnoma prilegal to"cnemu.
    \end{subsubsection}

    \begin{subsubsection}*{d)}
        Za preverjanje to"cnosti modela sem uporabil metodo zankanja oz. bootstrap. Glede na prvih par 
        podatkov, nimamo opravka z linearno funkcijo, tako da je napaka nekoliko ve"cja, kot v primeru b.
    \end{subsubsection}

\end{subsection}

\begin{subsection}*{Naloga 2}

    \begin{subsubsection}*{a)}
        Izri"sem nekaj standardnih statistik podatkov in jih standardiziram, da so vrednosti med seboj 
        bolj primerljive.
    \end{subsubsection}

    \begin{subsubsection}*{b)}
        To"cnosto modela ocenim s pomo"cjo pre"cnega preverjanja in izra"cunam 'accuracy score'.
        Nastavim 'random seed' na $42$ za ponovljivost rezultatov. Model je stabilen, saj pri $10$-kratnem
        prečnem preverjanju, 'accuracy score' ne preveč varira.
    \end{subsubsection}

    \begin{subsubsection}*{c)}
        Dodam spremenljivke, ki jih predlaga domenski ekspert in ponovno ocenim model s pomo"cjo 
        pre"cnega preverjanja. Tokrat se 'accuracy score'pove"ca iz $0.959$ na $0.987$. Glede na 
        koreliranost novih spremenljivk s starimi, sem posku"sal z odstranjevanjem starih spremenljivk, 
        ampak izbolj"sanje je bilo zanemarjljivo oz. ga ni bilo.
    \end{subsubsection}

    \begin{subsubsection}*{d)}
        Temu delu naloge sem se odlo"cil pristopiti na dva na"cina.
        \begin{enumerate}
            \item Pogledal sem absolutne vrednosti koeficientov po tem ko sem model ve"ckrat pognal in 
            obdr"zal koeficiente, ki so bili ve"cji kot 1 (seveda konstante nisem upo"steval kot koeficient).
            Tako sem obdr"zal le $4$ spremenljivke. Natan"cnost modela pa se prakti"cno ni spremenila, 
            saj je "ze izhodi"s"cna bila visoka. 
            \item Treniral sem modele kjer sem izvzel $i$-to spremenljivko in nato izra"cunal 'accuracy score' s 
            s pre"cnim preverjanjem. Nato sem izrisal graf spremembe natan"cnosti modela glede na izvzeto spremenljivko.
            Ni presenetljivo, da so na natan"cnost najbolj vpivale (skoraj vse) enake spremenljivke kot pri prvem pristopu. 
            Zanimivo je, da je izvzetje 2. spremenljivke najbolj izbolj"salo oceno modela, saj je imel 
            'accuracy score'nad $0.99$. 
        \end{enumerate}

        Na podlagi zgornjih pristopov lahko sklepamo, da so spremenljivke z indeksi 
        $0, 1, 4, 9$ najbolj pomembne za model, saj izvezetje ostalih ne povzro"ci vpada v 
        oceni modela oz. koeficient pred njimi so najvi"sji.
    \end{subsubsection}
\end{subsection}

\begin{subsection}*{Naloga 3}
    Nalgo sem za"cel s tem da sem odstranil spremenljivko 'zaporedna "stevilka diamanta v bazi', saj 
    o"citno ni vsebinsko povezana z njegovo ceno (Če bi premešali podatke bi se vpliv popolnoma spremenil)
    Standardiziral sem napovedne spremenljivke. Nato sem si vizualiziral porazdelitev cene diamantov in izpisal 
    Pearsonov korelacijski koeficient med napovednimi in odvisno spremenljivko, da sem dobil idejo kateri podatki 
    so korelirani in kateri ne.
    Nato sem izraisal grafe, ki prikazujejo odvisnoti napovednih in odvisne spremenljivke. Nisem ravno upo"steval grafov kategori"cnih spremenljivk, saj so nekoliko nesmiselni. Poskusil sem ustvariti raznovrstne napovedne spremenljivke, npr. opazil
    sem, da je odvisnost med stKaratov in ceno kvadratična in sem zato uvedel spremenljivko stKaratov$^2$, in "se marsikaj drugega.
    Nič ni dobro vplivalo na končno natančnost modelov, zato sem odločil, da ne bom dodal drugih napovednih 
    spremenljivk (Kot primer sem pustil volumen diamantov).
    Odstranil sem 'outlierje' iz podatkov, saj ti slabo vplivajo na splošnost modela.
    Kategori"cne spremenljivke sem spremenil v 'dummy variable'. Končno sem s 
    prečnim preverjanjem ocenil točnost knn regresijskega modela.
    Optimalen $k$ je 4 in RMSE pri tem ka je približno 650, kar sicer ni zelo dober rezultat (prakti"cno gledano), ampak moji opisani poskusi
    niso kaj dosti doprinesli k izboljšanju to"cnosti modela. $R^2$ je pribli"zno $0.984$, kar pomeni, da pojasnimo pribli"zno $98\%$ variabilnosti (Precej dober rezultat).
\end{subsection}


\end{document}