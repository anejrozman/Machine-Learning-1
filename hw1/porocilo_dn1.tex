\documentclass[a4paper, 10pt]{article}
\usepackage[utf8]{inputenc}
\usepackage[slovene]{babel}

\begin{document}
\title{Domača naloga 1 pri predmetu ITAP}
\author{Anej Rozman}
\date{}
\maketitle

\begin{subsection}*{Naloga 1}

    \begin{subsubsection}*{a)}
        Glede na to, da so podatki naklju"cni sem pri"cakoval, da bodo mogo"ce koeficienti bili 
        decimalna stevila. Dobil pa sem dve celi "stevili in dve "stevili, ki sta prakti"cno $0$. 
        \newline
        "Stevila: $[ 1,   5, -4.00027234\cdot 10^{-15}, \ -7.21644966\cdot10^{-16}]$
    \end{subsubsection}

    \begin{subsubsection}*{b)}
        Pri tem delu naloge sem samo dodal stolpec enic matriki $X$, saj glede na to da obravnavamo 
        podatke, ki so popolnoma linearno kolerirani med sabo, je to na"cin, da se bo napovedni model
        popolnoma prilegal to"cnemu.
    \end{subsubsection}

    \begin{subsubsection}*{d)}
        Za preverjanje to"cnosti modela sem uporabil metodo $5$-kratnega stratificiranega
         pre"cnega preverjanja, saj je v podatkovni mno"zici $2000$ primerov.
    \end{subsubsection}

\end{subsection}






\end{document}